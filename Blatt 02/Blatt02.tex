\documentclass{pi2}
\begin{document}

% Bei Gruppen mit weniger als drei Teilnehmenden 
% einfach die Gruppennummerklammer 
% und die entsprechenden TeilnehmerIn-Klammern leer lassen.
%
% z.B.: \maketitle{1}{20.11.2016}{Stefanie Schema}{}{Max Muster}{}{}

% \maketitle{Übungsnummer}{Abgabedatum}{TutorIn}{Gruppennummer}
%           {TeilnehmerIn 1}{TeilnehmerIn 2}{TeilnehmerIn 3}

\maketitle{2}{14.05.2017}{Nikhil Bhardwaj}{09}
          {Jana Göken}{Alex Rink}{Meira Iske}
\section{TheaPlusPlus}
\subsection{TheaPP}

Im Quellcode der Klasse TheaPP wurden die Methoden createInitialConfiguration(), distributeStudis(), nextConfiguration() und constraintsSatisfied() implementiert bzw. vervollständigt. In createinitialConfiguration() entsteht eine leere Startkonfiguration, indem ein leeres Feld von Listen erzeugt wird, die den Tutorien entsprechen, in die Später Studenten eingefügt werden können. In nextConfiguration() werden die vier Fälle umgesetzt, die in der Aufgabenstellung beschrieben sidn und eintreten können. Fälle 1-3 wurden von uns direkt implementiert, wobei wir für Fall vier die Hilfsmethode searchK() implementiert haben, welche das K sucht, ab dem alle Studenten aus den Tutorien entfernt werden. 
ContraintsSatisfied() unterscheidet eine gültige von einer ungültigen Konfiguration, indem die Bewertungen bezüglich der Mindestbewertungen der einzelnen Studenten überprüft werden. Sobald ein Student bezüglich der Mindestbewertung nicht in das Tutorium eingetragen werden kann, ist die Konfiguration nicht gültig. 
DistributeStudis() greift schließlich auf alle anderen Methoden zu und erzeugt eine möglichst gültige Verteilung der Studenten.



\begin{lstlisting}
public class TheaPP {

	/**
	 * Die maximal mögliche Bewertung von Studierenden für ein Tutorium. Die möglichen
	 * Werte für die Bewertung liegen damit zwischen 0 und diesem Wert (beide Werte
	 * inklusive).
	 */
	public static final int MAX_RATING = 5;

	/**
	 * Die aktuelle Konfiguration als Liste von Tutorien. Ein Tutorium wird hier als Liste
	 * von Studi-Objekten realisiert.
	 */
	private List<List<Studi>> tutorials;

	/**
	 * Die Anzahl der Tutorien.
	 */
	private final int noOfTutorials;

	/**
	 * Die maximale Anzahl von Studierenden in einem Tutorium.
	 */
	private final int maxTutorialSize;

	// TODO: evtl. eigene Attribute hinzufügen

	/**
	 * Erzeugt ein neues Objekt zur Verteilung von Studierenden auf die gegebene Anzahl
	 * von Tutorien mit der gegebenen maximalen TeilnehmerInnenzahl pro Tutorium.
	 *
	 * @param pNoOfTutorials
	 *          Die Anzahl der Tutorien.
	 * @param pMaxTutorialSize
	 *          Die maximale Anzahl an Studierenden pro Tutorium.
	 * @throws IllegalArgumentException
	 *           Falls die Anzahl der Tutorien oder die maximale Anzahl der Studierenden
	 *           pro Tutorium negativ ist.
	 */
	public TheaPP(final int pNoOfTutorials, final int pMaxTutorialSize) {
		// TODO: evtl. Code ergänzen
		if (pNoOfTutorials < 0) {
			throw new IllegalArgumentException("Number of tutorials must not be negative!");
		}
		if (pMaxTutorialSize < 0) {
			throw new IllegalArgumentException(
				"Maximum number of students in tutorial must not be negative!");
		}
		noOfTutorials = pNoOfTutorials;
		maxTutorialSize = pMaxTutorialSize;

		tutorials = createInitialConfiguration();
	}

	/**
	 * Erzeugt die Startkonfiguration, d.h. eine Liste von Tutorien mit der korrekten
	 * Größe. Ein Tutorium ist dabei eine Liste von Studi-Objekten mit der maximalen
	 * Anzahl an Studierenden plus Eins als Größe.
	 *
	 * @return die Startkonfiguration als Liste von Studi-Listen
	 */
	private List<List<Studi>> createInitialConfiguration() {
		List<List<Studi>> btutorials = new ArrayList<>(noOfTutorials);
		List<Studi> tutorial = new ArrayList<Studi>(maxTutorialSize+1);
		
		for(int i = 0; i<noOfTutorials; i++){
			btutorials.add(tutorial);
		}
		
		tutorials = btutorials;
		
		return tutorials;
		
	}

	/**
	 * Verteilt die Studierenden aus der gegebenen Studi-Liste auf die Tutorien unter
	 * Berücksichtigung des gegebenen Wertes für die Mindestbewertung. Das bedeutet, dass
	 * Studierende niemals einem Tutorium zugeordnet werden dürfen, das sie mit einer
	 * Bewertung niedriger als die Mindestbewertung bewertet haben. Gleichzeitig dürfen
	 * pro Tutorium höchstens so viele Studierende zugeordnet werden, wie zuvor durch den
	 * Konstruktorparameter festgelegt (allgemein auf keinen Fall mehr als
	 * {@link TheaPP#MAX_STUDENTS_IN_TUTORIAL}).
	 *
	 * Wenn eine Verteilung der gegebenen Studierenden auf die Tutorien mit der gegebenen
	 * Mindestbewertung möglich und durchgeführt worden ist, wird {@code true}
	 * zurückgegeben, ansonsten {@code false}.
	 *
	 * @param pStudis
	 *          Die Liste mit den zu verteilenden Studis.
	 * @param pMinRating
	 *          Die Mindestbewertung für die Verteilung auf die Tutorien.
	 * @return {@code true} falls eine Verteilung möglich war, ansonsten {@code false}.
	 * @throws IllegalArgumentException
	 *           Falls die gegebene Mindestbewertung kleiner als 1 oder größer als
	 *           {@link TheaPP#MAX_RATING} ist, die Liste der zu verteilenden Studis
	 *           {@code null} ist oder nicht genug Platz in den Tutorien für die Studis
	 *           ist.
	 */
	public final boolean distributeStudis(final List<Studi> pStudis,
		final int pMinRating) {
		if (pMinRating < 1 || pMinRating > TheaPP.MAX_RATING) {
    		throw new IllegalArgumentException("invalid minimum rating");
    	}
    	lastStudi =pStudis.get(0);
    	tutorials.get(0).add(lastStudi);
    	lastStudi.setTutorial(0);
    	int j = 0;
    	while (!(constraintsSatisfied(tutorials, pMinRating))) {
    		tutorials.get(j).remove(lastStudi);
    		tutorials.get(j+1).add(lastStudi);
    		lastStudi.setTutorial(j+1);
    		j++;
    	}
    	int counter = 0;
    	while (counter != -1) {
    		counter = nextConfiguration(tutorials, pStudis, counter, pMinRating);
    		if (counter == pStudis.size()) {
    			return true;
    		}
    	}
        resetConfiguration();
        return false;
	}

	/**
	 * Berechnet die Folgekonfiguration der gegebenen Konfiguration für die gegebene Liste
	 * von Studierenden, den Index des zuletzt zugeordneten Studis der gegebenen
	 * Konfiguration und die Mindestbewertung. Die Berechnung der Folgekonfiguration
	 * geschieht "in-place", d.h. die gegebene Konfiguration wird direkt durch diese
	 * Methode verändert. Gibt den Index des bei der Berechnung der Folgekonfiguration
	 * zuletzt zugeordneten Studis zurück.
	 *
	 * Die Methode ist nicht private, sondern package-private, da sie durch JUnit-Tests
	 * getestet werden soll. Da das Paket vor einer Auslieferung versiegelt wird, ist sie
	 * damit von außen nicht aufrufbar. Aus diesem Grund werden die Parameterwerte auch
	 * nicht auf sinnvolle Werte überprüft. Insbesondere darf diese Methode nicht mit
	 * einer Konfiguration aufgerufen werden, für die es keine Folgekonfiguration gibt.
	 *
	 * @param pConfiguration
	 *          Die Konfiguration, deren Folgekonfiguration errechnet werden soll.
	 * @param pStudents
	 *          Die Liste der Studierenden.
	 * @param pLastStudiIndex
	 *          Der Index der zuletzt zugeordneten StudentIn der gegebenen Konfiguration.
	 * @param pMinRating
	 *          Die Mindestbewertung.
	 * @return Den Index des zuletzt zugeordneten Studis oder -1 wenn es keine
	 *         Folgekonfiguration gibt und nicht die letzte StudentIn zugeordnet wurde
	 *         oder die Anzahl der StudentInnen wenn alle StudentInnen erfolgreich
	 *         zugeordnet wurden.
	 */
	final int nextConfiguration(final List<List<Studi>> pConfiguration,
		final List<Studi> pStudents, final int pLastStudiIndex, final int pMinRating) {
		throw new UnsupportedOperationException(); // TODO: implementieren
		
		//Fall 1
		if(constraintsSatisfied(pConfiguration, pMinRating) && (pLastStudiIndex == pStudents.size())){
			
		}
		//Fall 2
		if(constraintsSatisfied(pConfiguration, pMinRating) && (pLastStudiIndex < pStudents.size())){
			pConfiguration.get(1).add(pStudents.get(pLastStudiIndex+1));
		}
		//Fall 3 ???
		if(!this.constraintsSatisfied(pConfiguration, pMinRating) && (pConfiguration.indexOf(pLastStudiIndex) < noOfTutorials)) { // && j < t 
	         int j;
	          
	          for(int i = noOfTutorials; i>-1; i--){
					if( (pConfiguration.get(i)).contains(pStudents.get(pLastStudiIndex)) ){
						 j = i;
					}
	          }
	          
			 pConfiguration.get(j).remove(pStudents.get(pLastStudiIndex));
			 pConfiguration.get(j+1).add(pStudents.get(pLastStudiIndex));
			 
	          
	          
		}
		//Fall 4
		else {
				k = this.searchK(pStudents,pConfiguration);
				if(k < 0){
					this.resetConfiguration();
					return -1;
				}
				else{
					for(int i = k+1, i <= pLastStudiIndex, i++){
						(pConfiguration.get(noOfTutorials-2)).remove(pStudents.get(i));
					}
					for(int i = noOfTutorials-2, i > -1 , i--){
							if( (pConfiguration.get(i)).contains(pStudents.get(k)) ){
								(pConfiguration.get(i)).remove(pStudents.get(k));
							}
					}
				(pConfiguration.get(i+1)).add(pStudents.get(k));
				tutorials = pConfiguration;
				return k
	}

	/**
	 * Prüft, ob die gegebene Konfiguration hinsichtlich der gegebenen Mindestbewertung
	 * gültig ist. Wenn das so ist, wird {@code true} zurückgegeben, sonst {@code false}.
	 *
	 * Die Methode ist nicht private, sondern package-private, da sie durch JUnit-Tests
	 * getestet werden soll. Da das Paket vor einer Auslieferung versiegelt wird, ist sie
	 * damit von außen nicht aufrufbar.
	 *
	 * @param pConfiguration
	 *          Die zu prüfende Konfiguration als Liste von Studi-Listen.
	 * @param pMinRating
	 *          Die Mindestbewertung der Studis für ihre Tutorien.
	 * @return {@code true} falls kein Tutorium mehr als die erlaubte Maximalzahl von
	 *         Studis enthält und kein Studi einem Tutorium zugeordnet ist, für das sie
	 *         oder er eine geringere Bewertung als die gegebene Mindestbewertung vergeben
	 *         hat, ansonsten {@code false}.
	 */
	final boolean constraintsSatisfied(final List<List<Studi>> pConfiguration,
		final int pMinRating) {
		
       
		for(int i= 0; i<noOfTutorials; i++) {
		 for(int j = 0; j<maxTutorialSize; j++){
			 
		  Studi student = pConfiguration.get(i).get(j);
			 if(student.getRating(i) < pMinRating){
				 return false;
			 }
		 }
		}
		return true;
	}

	/**
	 * Setzt die Konfiguration auf die Startkonfiguration zurück.
	 */
	private void resetConfiguration() {
		for (final List<Studi> tutorial : tutorials) {
			tutorial.clear();
		}
	}

	/**
	 * Gibt die aktuelle Verteilung der Studierenden in der Konsole aus. Dabei steht in
	 * jeder Zeile ein Tutorium und die Teilnehmenden werden mit ihrem Account und der
	 * Bewertung für ihr aktuell zugeordnetes Tutorium durch Komma getrennt aufgeführt.
	 */
	private void printDistribution() {
		int tutNo = 1;
		for (final List<Studi> tutorial : tutorials) {
			System.out.print(String.format("Tut %02d:", tutNo));
			System.out.println(Arrays.toString(tutorial.toArray()));
			tutNo++;
		}
	}
	
	/**
	 * Hilfsmethode
	 * Sucht das k' aus Fall 4 und gibt dieses zurück, falls es eines gibt.
	 * Ansonsten wird -1 zurückgegeben.
	 */
	private int searchK(List<Studi> pStudents, List<List<Studi>> pConfiguration, int pLastStudiIndex){
		int k = -1;
		
		for(int i = noOfTutorials-2; i > -1; i--){
			for(int j = 0; j < maxTutorialSize; j++){
				if(
					pConfiguration.get(i)).get(j) != null
					&& k < pStudents.indexOf( ((pConfiguration.get(i)).get(j))
					&& for( int l = k+1, l <= pLastStudiIndex, l++) {
							(pConfiguration.get(noOfTutorials-1)).contains(pStudents.get(l));
					   }
					){
						k = pStudents.indexOf( ((pConfiguration.get(i)).get(j)) );
				}
			}
		}
		return k;
	}
	
\end{lstlisting}

\subsection{Main}

Für die Main-Methode haben wir zunächst die Hilfsmethode validSetup angelegt, die die in der JavaDoc beschrieben Funktion hat. Die Methode war leider nicht ausfürbar, da es an anderer Stelle Probleme zu geben scheint, die Main-Methode an sich sollte jedoch funktionstüchtig sein.

\begin{lstlisting}

	/**
	 * Hilfmethode. Dient zur Überpruefung der Parameter, also ob
	 * es ausreichend viele Plätze in den Tutorien für alle Studenten gibt und ob jeder
	 * Student mindestens ein Tutorium mit dem Minimum-Rating bewertet hat. In diesem Fall
	 * wird {@code true} zurueckgegeben. 
	 * Ist eine dieser beiden Bedingungen verletzt, wird {@code false} zurueckgegeben.
	 * 
	 * @param pStudis
	 * @param pNoOfTutorials
	 * @param pMaxTutorialSize
	 * @param pMinRating
	 * @return
	 */
	private boolean validSetup(final List<Studi> pStudis, final int pNoOfTutorials,
			final int pMaxTutorialSize, final int pMinRating) {
			if (pStudis.size() > pNoOfTutorials * pMaxTutorialSize) {
			return false; }
			boolean hasMinRat;
			for (Studi studi :pStudis) {
			hasMinRat = false;
			for (int k = 0 ; k < pNoOfTutorials ; k++) {
				if (studi.getRating(k) >= pMinRating) {
    				hasMinRat = true;
    				break;
    				}
    			}
    		if (hasMinRat) {
    			continue;
    		}
    		return false;
		}
    	return true;
    	}
    	
    		/**
	 * Liest die Dateien FiveTutorials.csv, SevenTutorials.csv und
	 * TenTutorials.csv ein, parst sie und versucht die Studierenden auf die Tutorien zu
	 * verteilen. Wenn die Verteilung gelingt, wird sie auf der Konsole ausgegeben.
	 *
	 * @param args
	 *          Werden ignoriert.
	 * @throws TheaPPParseException
	 *           Falls eine der Dateien nicht eingelesen werden kann.
	 */
	public static void main(final String[] args)  {
		final String filename = args[0];
    	final int noOfTutorials = Integer.parseInt(args[1]);
    	final int maxTutorialSize = Integer.parseInt(args[2]);
    	final int minRating = Integer.parseInt(args[3]);
    	final TheaPP thea = new TheaPP(noOfTutorials, maxTutorialSize);
    	try {
    		final List<Studi> studiList = (new TheaPPParser(noOfTutorials )).parseStudents(filename);
    		if(!(thea.validSetup(studiList, noOfTutorials, maxTutorialSize, minRating))) {
    			throw new IllegalArgumentException("invalid setup");
    		}
    		if (thea.distributeStudis(studiList, minRating)) {
    			thea.printDistribution();
    		}
    		else {
    			throw new IllegalArgumentException("ivalid setup");
    		}
    	}
    		catch (Exception e) {
    			System.out.println(e);
    		}
    	}
	}
\end{lstlisting}
\section{Bonus}

Aufgrund der Probleme bei der Ausführung der Main-Methode erwies es sich als äußerst schwierig, Vermutungen zu den angeführten Fragestellungen aufzustellen. 

Es könnte sich als problematisch erweisen, dass einige der Studenten die Mindestbewertung von 4 für kein Tutorium angegeben haben und deshalb nie alle Studenten zufriedengestellt verteilt sind.
\end{document}

